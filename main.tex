\documentclass[aspectratio=169]{beamer}
\usepackage{pgfpages}
\setbeamertemplate{note page}[plain]

%enable notes
%\setbeameroption{show notes on second screen=right}
\usepackage[T1]{fontenc}
\usepackage[utf8]{inputenc}
\usepackage{lmodern}
\usepackage[french]{babel}
\usepackage{hyperref}
\usepackage{graphicx}
\usepackage{makecell}
\usepackage{xcolor}
\usepackage{colortbl}
\usepackage[flushleft]{threeparttable}
\setbeamertemplate{note page}{\pagecolor{yellow!5}\insertnote}\usepackage{palatino}

\graphicspath{ {./img/} }

\newcommand{\TODO}{TODO:}
\newcommand*{\rot}{\rotatebox{90}}


\title{Redis\\~\\AIT --- Présentation}

% \titlegraphic{\includegraphics [height=.2\textheight] {logo_bacula.png}}
\author{Cassandre Wojciechowski \\ Gwendoline Dössegger \\ Noémie Plancherel \\ Gaby Roch}
\hypersetup{pdfauthor={G. Dössegger, N. Plancherel, G. Roch, C. Wojciechowski}}
\date{\today}

\begin{document}

\begin{frame}
  \titlepage
\end{frame}

\begin{frame}{Redis -- Présentation générale}
  \TODO
 \begin{itemize}
  \item Logiciel de sauvegarde multi-platform créé en 2000 par Kern Sibbald
  \item Première version 2002
  \item Dernière version 11.0.5, juin 2021
  \item Open source / private source
 \end{itemize}
\note{Bacula est un logiciel de sauvegarde multi-platforme. Il a été développé à partir de l'année 2000 par Kern Sibbald et sa première version a été publiée deux ans plus tard, c'est-à-dire en 2002. 
\\~\\
Le logiciel est continuellement maintenu, documenté et mis à jour. Pour vous donnez une idée, la dernière release est sortie en juin de cette année. 
\\~\\
Bacula possède une version community qui est Open Source. Ses différents composants peuvent tourner sur des systèmes d'exploitation gratuit (par ex linux) ce qui ne nécessite pas l'achat de licences supplémentaires pour tester le logiciel.
\\~\\
Il existe une version entreprise avec plus de fonctionnalités mais cette fois-ci payante que l'on vous présentera par la suite.}
\end{frame}

\begin{frame}{Architecture}
 \note{ Bacula est composé de 5 composants principaux : }
\end{frame}

\begin{frame}{Architecture - Chiffrement}
 \begin{itemize}
  \item Chiffrement et signature des backups possible
  \item La master key doit être sauvegardée manuellement hors-site
  \item S'il y a perte de la master key, les backups sont perdus
 \end{itemize}
\end{frame}

\begin{frame}{Démonstration}
\end{frame}

\begin{frame}{CPU}
\end{frame}

\begin{frame}{Coûts}
\begin{center}

\begin{itemize}
    \item 2 versions: Community Open Source et Entreprise
    \item La version Entreprise propose (entre autres) un expert qui permet d'apporter un support technique
    \item Dès 500\$ par mois sur devis pour la version Entreprise (abonnement Standard)
\end{itemize}

 \begin{tabular}{|l|ccccc|}
    \hline
    & Standard & Bronze & Silver & Gold & Platinum \\
    \hline
    \hline
    Nombre de clients & < 50 & < 200 & < 500 & < 2000 & < 5000 \\
    \hline
    Nombre de contacts & 1 & 2 & 3 & 5 & 5 \\
    \hline
    Assistance & Web & Web & Web et tél.& Web et tél. & Web et tél. \\
    \hline
    Nombre d'OS & 4 & toutes & toutes & toutes & toutes \\
    \hline
    Temps de réponse & 1j -- 4j & 6h -- 4j & 4h -- 2j & 1h -- 2j & 1h -- 2j \\ %heure/jour ouvrable
    du support &  & & & & \\
    \hline
    Formation & 0 & 0 & 0 & 0 & 1/an \\
    \hline
 \end{tabular}
\end{center}
\end{frame}


\begin{frame}{Comparaison}
 \begin{center}
  \begin{tabular}{|l||c|c|c|c|c|c|c|c|c|c|}
    \hline
    & Coût & \rot{Open source ~} & Backups & \rot{Déduplication ~} & \rot{Chiffrement ~} & \rot{Compression ~} & \rot{Web interface} & \rot{Linux} & \rot{MacOS X} & \rot{Windows} \\
    \hline
    \hline
    Bacula & \makecell{Gratuit\\Payant} & \cellcolor{green!50} & \makecell{Full\\ Incrémentaux\\ Différentiels} & \cellcolor{green!50} & \cellcolor{green!50} & \cellcolor{green!50} & \cellcolor{green!50} & \cellcolor{green!50} & \cellcolor{green!50} & \cellcolor{green!50} \\
    \hline
    BorgBackup & \makecell{Gratuit\\Payant} & \cellcolor{green!50} & \makecell{Full\\ Incrémentaux\\ Différentiels} & \cellcolor{green!50} & \cellcolor{green!50} & \cellcolor{green!50} & \cellcolor{orange!50} & \cellcolor{green!50} & \cellcolor{green!50} & \cellcolor{red!50} \\
    \hline
    Bup & Gratuit & \cellcolor{green!50} & Incrémentaux & \cellcolor{green!50} & \cellcolor{red!50} & \cellcolor{green!50} & \cellcolor{green!50} & \cellcolor{green!50} & \cellcolor{green!50} & 
    \cellcolor{red!50} \\
    \hline
    Duplicati & Gratuit & \cellcolor{green!50} & \makecell{Full\\ Incrémentaux} & \cellcolor{green!50} & \cellcolor{green!50} & \cellcolor{green!50} & \cellcolor{green!50} & \cellcolor{green!50} & \cellcolor{green!50} & \cellcolor{green!50} \\
    \hline
    Rsync & Gratuit & \cellcolor{green!50} & Incrémentaux & \cellcolor{red!50} & \cellcolor{red!50} & \cellcolor{red!50} & \cellcolor{red!50} & \cellcolor{green!50} & \cellcolor{green!50} & \cellcolor{green!50}  \\
    \hline
 \end{tabular}
 \begin{tablenotes}
 \scriptsize 
  \fcolorbox{black}{green!50}{\rule{0pt}{4pt}\rule{4pt}{0pt}}\quad Oui \fcolorbox{black}{red!50}{\rule{0pt}{4pt}\rule{4pt}{0pt}}\quad Non
  \fcolorbox{black}{orange!50}{\rule{0pt}{4pt}\rule{4pt}{0pt}}\quad En développement
    \end{tablenotes}
 \end{center}
\end{frame}

\begin{frame}{Avantages / inconvénients}
    \begin{center}\small
     \begin{tabular}{|l|l|}
     \hline
      \textbf{Avantages} & \textbf{Inconvénients} \\
     \hline
     \hline
        Interface utilisateur graphique & Prise en main complexe (fichiers \\ 
                                        & de configuration) \\
     \hline
     Cross-platform & Pas de scan de virus \\
     \hline
     Automatisation \& planification des tâches & Configuration nécessaire au backup \\ 
                                                & des appareils mobiles\\
     \hline
     Sauvegarde continue (CDP) & \\
     \hline
     Toujours maintenu (dernière update : juin 2021) & \\
     \hline
     Sauvegarde Cloud \& hors-site & \\
     \hline
     Charge réseau \& CPU modulables & \\
     \hline
     Déduplication des données & \\
     \hline
     Support d'environnements virtuels & \\
     \hline
     Large documentation et en plusieurs langues & \\
     \hline
     \end{tabular}

    \end{center}
\end{frame}

\begin{frame}{Conclusion}
 \begin{itemize}
  \item Sauvegarde jusqu'à 1x par heure - RPO très faible
  \item Automatisation et planification possibles
  \item Economie de temps, d'argent, de stress
 \end{itemize}
 \note{
 Bacula est très flexible et les sauvegardes peuvent être planifiées au bon vouloir de l'administrateur, cela signifie que des backups peuvent être effectués jusqu'à une fois par heure. Cela nous donne un RPO (Recovery Point Objective) très faible. Ce RPO dépend de la manière dont Bacula est configuré. 
\\~\\
Le RPO très faible et l'automatisation possible des sauvegardes engendrent une réelle économie pour l'entreprise, on économise le temps des employés et leur travail qu'on ne risque pas de trop perdre.
}
\end{frame}

\begin{frame}{Questions ?}
\end{frame}


\end{document}
